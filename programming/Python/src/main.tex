\documentclass[a4paper]{article}
\usepackage[utf8]{inputenc}


%=-=-=-=-=-=-=-=-=-=-=-=-=-=-=-=-=-=-=-=-=-=-=-=-=-=-=-=-=-=-=-=-=-=-=-=-=-=-=-=-
% PREAMBLE
%=-=-=-=-=-=-=-=-=-=-=-=-=-=-=-=-=-=-=-=-=-=-=-=-=-=-=-=-=-=-=-=-=-=-=-=-=-=-=-=-

%%%%%%%%%%%%%%%%%%%%%%%%%%%%%%%%%%%%%%%%%%%%%%%%%%%%%%%%%%%%%%%%%%%%%
% Important styling notes
%%
% For now, to include img.jpg in img/path/to/img.jpg, just use:
% path/to/img.jpg - for details see style.tex
%=-=-=-=-=-=-=-=-=-=-=-=-=-=-=-=-=-=-=-=-=-=-=-=-=-=-=-=-=-=-=-=-=-=-=-=-=-=-=-=-
% Packages
%%
%\usepackage{fullpage} % Package to use full page
\usepackage[top=1in,bottom=1in,left=1in,right=1in,heightrounded]{geometry}

\usepackage{parskip}                    % Package to tweak paragraph skipping
\usepackage{amsmath}                    % standard
\usepackage{amssymb}                    % standard - Double R symbol etc.
\usepackage{hyperref}
\usepackage{amsthm}                     % standard - theorem, definition, etc.
\usepackage{multicol}                   % multiple columns for numbering
\usepackage{enumitem}                   % standard - enumerate styles
\usepackage[utf8]{inputenc}
\usepackage{scrextend}                  % indentation
\usepackage{graphicx}                   % standard - add figures
\usepackage{float}                      % standard - figure position, use [H] option
\usepackage{pifont}                     % symbols http://willbenton.com/wb-images/pifont.pdf
                                        % e.g. \ding{51}
\usepackage{gensymb}                    % degree symbol \degree
\usepackage{xcolor}                     % bg color
\hypersetup{
    colorlinks,
    linkcolor={black!50!black},
    citecolor={blue!50!black},
    urlcolor={blue!80!black}
}
\usepackage{framed}                     % bg color
\usepackage[T1]{fontenc}                % small caps
\usepackage{sectsty}                    % headings colour
\usepackage{mathtools}                  % Loads amsmath
\usepackage{amsthm,thmtools,xcolor}     % coloured theorem
\usepackage[toc,page]{appendix}         % reference to appendix
%\usepackage{titlesec}                   % change chapter, section, etc. formats
\usepackage{xifthen}                    % if, else
\usepackage{etoolbox}
% format numbering in theorem, lemma, etc. environment
\AtBeginEnvironment{theorem}{\setlist[enumerate, 1]{font=\upshape,  wide=0.5em, before=\leavevmode}}
\AtBeginEnvironment{lemma}{\setlist[enumerate, 1]{font=\upshape,  wide=0.5em, before=\leavevmode}}
\usepackage[letterspace=150]{microtype} % \textls{<letterspaced text>} % 0 <= letterspace <= 1000, 1000 = M space
\usepackage{letltxmacro}                % renew commands?
\usepackage{minted}                     % package to list code
    % otherwise minted goes off the page
    \setmintedinline{breaklines}
\usepackage{subfig}
\usepackage{eso-pic}                    % title page bg pic
\usepackage{varwidth}
\PassOptionsToPackage{svgnames}{xcolor}
\usepackage{fontawesome}                % \faQuestionCircle
\usepackage{marvosym}                   %\Pointinghand
\usepackage{mdframed}                   % easy outline frames
\usepackage[many]{tcolorbox}            % colour box for theorem styles
\usepackage{array,booktabs,calc} % table figs and text
\usepackage{comment}                    % \begin{comment}
\usepackage{fancyhdr}                   % page headings
\usepackage{mdframed}                   % boxes
\usepackage[backend=biber,sorting=none,style=ieee]{biblatex}
\usepackage{caption}
%%% caption options {
%\DeclareCaptionFont{white}{\color{white}}
\DeclareCaptionFormat{listing}{\colorbox{magenta!30!gray}{\parbox{\textwidth}{#1#2#3}}}
\captionsetup[lstlisting]{format=listing,labelfont={bf,small},textfont=small,skip=-1pt}
%%% }
\addbibresource{bibliography.bib}
\usepackage{url}
\usepackage{textcomp}
\usepackage[makeroom]{cancel}           % crossed symbols - \cancel{}, \bcancel{}, xcancel{}
\usepackage{algorithm}
\usepackage[noend]{algpseudocode}
\usepackage{tikz}
\usetikzlibrary{arrows.meta,positioning,quotes} % arrows and nodes in tikz
\usepackage{marginnote}                 % things in page margin by \marginnote{...}
\usepackage{pgfplots}
\usepackage{pstricks-add,pst-slpe}      % for fancy tikz arrows
%\usepackage{titlesec}                  % title style
\usepackage{lmodern}                    % a font
\usepackage{titletoc}                   % Required for manipulating the table of contents
\usepackage{titlesec}                   % Allows customization of titles
\usepackage{fouriernc}                  % Use the New Century Schoolbook font
\usepackage{booktabs}                   % better tables
\usepackage{stmaryrd }                  % \varoast
\usepackage{listings}                   % code listings
\usepackage{longtable}                  % table across multiple pages
\usepackage{todonotes}                  % TODO bubbles by \todo{...} command
\usepackage{changepage}                 % paragraph margins
\usepackage{tikz}
\usetikzlibrary{calc}
\usepackage{eso-pic}
\usepackage{transparent}
\usepackage[makeroom]{cancel}

%=-=-=-=-=-=-=-=-=-=-=-=-=-=-=-=-=-=-=-=-=-=-=-=-=-=-=-=-=-=-=-=-=-=-=-=-=-=-=-=-
% Colours for various things
%%


\definecolor{shadecolor}{rgb}{1.,0.933,0.96} % bg color, r,g,b <= 1
\definecolor{medium_blue}{RGB}{60,125,190}
\definecolor{dark_blue}{RGB}{25,60,85}
\definecolor{dark_red}{RGB}{77,16,16}
\definecolor{LightPink}{rgb}{0.92.,0.8,0.84} % bg color, r,g,b <= 1
\definecolor{LighterPink}{rgb}{1.,0.94,0.97} % bg color, r,g,b <= 1
\definecolor{LightestPink}{rgb}{1.,0.95,0.99} % bg color, r,g,b <= 1
\definecolor{DarkestPink}{rgb}{0.36, 0.0, 0.18}
\definecolor{DarkerPink}{rgb}{0.41, 0.0, 0.21}
\definecolor{DarkPink}{rgb}{0.55, 0.05, 0.37}
\definecolor{lightestestpink}{RGB}{255,248,252}
\definecolor{codegray}{rgb}{0.5,0.5,0.5}
\definecolor{codegrayblue}{rgb}{0.35,0.35,0.47}



%=-=-=-=-=-=-=-=-=-=-=-=-=-=-=-=-=-=-=-=-=-=-=-=-=-=-=-=-=-=-=-=-=-=-=-=-=-=-=-=-
% Define my own theorem styles
%%

% "base" styles
\declaretheoremstyle[
  headfont=\color{DarkPink}\bfseries,
  bodyfont=\itshape,
]{colored}

\declaretheoremstyle[
  headfont=\color{DarkPink}\bfseries,
  bodyfont=\normalfont,
]{colored_upright}

% theorems (corollaries, etc) themselves, inherit from my style above
% Usage:
% \begin{theorem} \end{theorem}, \begin{lemma} \end{lemma}, ...
\declaretheorem[
	numberwithin=section,
 	style=colored,
	name=\textsc{Theorem},
]{theorem}

\tcolorboxenvironment{theorem}{
  boxrule=0pt,
  boxsep=2pt,
  colback={magenta!25!white},
  colframe=DarkPink,
  enhanced jigsaw, 
  borderline west={2pt}{0pt}{DarkPink},
  sharp corners,
  before skip=5pt,
  after skip=5pt,
  breakable,
  right=0mm % for equations
}

\declaretheorem[
	numberwithin=section,
 	style=colored,
	name=\textsc{Corollary},
]{corollary}

\tcolorboxenvironment{corollary}{
  boxrule=0pt,
  boxsep=1pt,
  colback={magenta!10!white},
  colframe=DarkPink,
  enhanced jigsaw, 
  borderline west={2pt}{0pt}{DarkPink},
  sharp corners,
  before skip=5pt,
  after skip=5pt,
  breakable,
  right=0mm % for equations
}

\declaretheorem[
	numberwithin=section,
	style=colored,
	name=\textsc{Lemma},
]{lemma}

\tcolorboxenvironment{lemma}{
  boxrule=0pt,
  boxsep=1pt,
  colback={magenta!10!white},
  colframe=DarkPink,
  enhanced jigsaw, 
  borderline west={2pt}{0pt}{DarkPink},
  sharp corners,
  before skip=5pt,
  after skip=5pt,
  breakable,
  right=0mm % for equations
}

\declaretheorem[
	numberwithin=section,
	style=colored,
	name=\textsc{Definition},
]{definition}

\tcolorboxenvironment{definition}{
  boxrule=0pt,
  boxsep=1pt,
  colback={magenta!25!white},
  colframe=DarkPink,
  enhanced jigsaw, 
  borderline west={2pt}{0pt}{DarkPink},
  sharp corners,
  before skip=5pt,
  after skip=5pt,
  breakable,
  right=0mm % for equations
}

\declaretheorem[
	numberwithin=section,
  	style=colored,
  	name=\textsc{Example},
]{exmp}

\declaretheorem[
	numberwithin=section,
  	style=colored,
  	name=\textsc{Solution},
]{soln}

%%% code listings
\lstdefinestyle{code1}{
    backgroundcolor=\color{lightestestpink},   
    commentstyle=\color{codegrayblue},
    keywordstyle=\color{DarkerPink},
    numberstyle=\tiny\color{codegray},
    stringstyle=\color{black!40!cyan},
    basicstyle=\small\ttfamily,
    breakatwhitespace=false,
    breaklines=true,        
    captionpos=t,             
    keepspaces=true,        
    numbers=left,           
    numbersep=5pt,
    showspaces=false, 
    showstringspaces=false,
    showtabs=false,
    tabsize=4
}

%%% code listings
\lstdefinestyle{code1}{
    backgroundcolor=\color{lightestestpink},   
    commentstyle=\color{codegrayblue},
    keywordstyle=\color{DarkerPink},
    numberstyle=\tiny\color{codegray},
    stringstyle=\color{black!40!cyan},
    basicstyle=\small\ttfamily,
    breakatwhitespace=false,
    breaklines=true,        
    captionpos=t,             
    keepspaces=true,        
    numbers=left,           
    numbersep=5pt,
    showspaces=false, 
    showstringspaces=false,
    showtabs=false,
    tabsize=4
}


\lstdefinestyle{terminal}{
    backgroundcolor=\color{black!5},   
    commentstyle=\color{codegrayblue},
    keywordstyle=\color{DarkerPink},
    %numberstyle=\tiny\color{codegray},
    stringstyle=\color{black!40!cyan},
    basicstyle=\small\ttfamily,
    numbers=none,
    breakatwhitespace=false,
    breaklines=true,        
    %captionpos=t,             
    keepspaces=true,        
    %numbers=left,           
    %numbersep=5pt,
    showspaces=false, 
    showstringspaces=false,
    showtabs=false,
    tabsize=4
}

\lstset{style=code1}

%=-=-=-=-=-=-=-=-=-=-=-=-=-=-=-=-=-=-=-=-=-=-=-=-=-=-=-=-=-=-=-=-=-=-=-=-=-=-=-=-
% Headers (size, font, colour)
%%

\makeatletter
\renewcommand{\@seccntformat}[1]{\llap{\textcolor{DarkestPink}{\csname the#1\endcsname}\hspace{1em}}}                    
\renewcommand{\section}{\@startsection{section}{1}{\z@}
{-4ex \@plus -1ex \@minus -.4ex}
{1ex \@plus.2ex }
{\normalfont\large\sffamily\bfseries\textcolor{DarkestPink}}}
\renewcommand{\subsection}{\@startsection {subsection}{2}{\z@}
{-3ex \@plus -0.1ex \@minus -.4ex}
{0.5ex \@plus.2ex }
{\normalfont\sffamily\bfseries\textcolor{DarkestPink}}}
\renewcommand{\subsubsection}{\@startsection {subsubsection}{3}{\z@}
{-2ex \@plus -0.1ex \@minus -.2ex}
{.2ex \@plus.2ex }
{\normalfont\small\sffamily\bfseries\textcolor{DarkestPink}}}                        


%=-=-=-=-=-=-=-=-=-=-=-=-=-=-=-=-=-=-=-=-=-=-=-=-=-=-=-=-=-=-=-=-=-=-=-=-=-=-=-=-
% Numberings, counters and spacings
%%
\numberwithin{equation}{section} % section number in eq/s
\setlength{\jot}{7pt} % spacing in split, gathered env/s



%% Custom examples
%% Output - Example 1,2,...
\newcounter{example}
\newenvironment{example}[1][]{\refstepcounter{example}\par\medskip
   \textbf{Example~\theexample. #1} \rmfamily}{\medskip}
%%%%%%%%%%%% End of unused %%%%%%%%%%%%



%=-=-=-=-=-=-=-=-=-=-=-=-=-=-=-=-=-=-=-=-=-=-=-=-=-=-=-=-=-=-=-=-=-=-=-=-=-=-=-=-
% Paths
%%

%=-=-=-=-=-=-=-=-=-=-=-=-=-=-=-=-=-=-=-=-=-=-=-=-=-=-=-=-=-=-=-=-=-=-=-=-=-=-=-=-
% User defined macros (math mode)
%%


% Curly braces under text. Usage: \myunderbrace{upper}{lower}
\newcommand{\myunderbrace}[2]{\mathrlap{\underbrace{\phantom{#1}}_{#2}} #1}
\newcommand{\setR}{\mathbb{R}} % \ouble R
\newcommand{\setRn}{\mathbb{R}^n} %  double R^n
\newcommand{\setN}{\mathbb{N}} % double N
\newcommand{\setZ}{\mathbb{Z}} % double Z
\let\oldemptyset\emptyset
\let\emptyset\varnothing % nice - looking empty set symbol
\newcommand{\fancyN}{\mathcal{N}} % null space
\newcommand{\fancyR}{\mathcal{R}} % range

\newcommand{\ba}{\textbf{a}}
\newcommand{\be}{\textbf{e}}
\newcommand{\bw}{\textbf{w}}
\newcommand{\bx}{\textbf{x}}
\newcommand{\bu}{\textbf{u}}
\newcommand{\bv}{\textbf{v}}
\newcommand{\by}{\textbf{y}}
\newcommand{\bz}{\textbf{z}}
\newcommand{\bb}{\textbf{b}}
\newcommand{\bA}{\textbf{A}}
\newcommand{\bB}{\textbf{B}}
\newcommand{\bC}{\textbf{C}}
\newcommand{\bD}{\textbf{C}}
\newcommand{\bI}{\textbf{I}}
\newcommand{\bM}{\textbf{M}}
\newcommand{\bO}{\textbf{0}}
\newcommand{\bS}{\textbf{S}}
\newcommand{\bX}{\textbf{X}}
\newcommand{\bU}{\textbf{U}}
\newcommand{\bY}{\textbf{Y}}
% double bars as in norm
%\newcommand{\norm}[1] {\left|\left| #1 \right| \right|} 
\newcommand{\norm}[1]{\left\lVert#1\right\rVert}
\renewcommand{\t}{^{\top}}

\newcommand{\mean}[1]{\bar{#1}}
\newcommand{\var}{\sigma^2}

\newcommand{\partdevx}[1]{\frac{\partial #1}{\partial x}}
\newcommand{\partdevt}[1]{\frac{\partial #1}{\partial t}}
\newcommand{\partdevxx}[1]{\frac{\partial #1}{\partial x}}
\newcommand{\partdevxn}[1]{\frac{\partial^n #1}{\partial x^n}}
\newcommand{\partdevy}[1]{\frac{\partial #1}{\partial y}}
\newcommand{\partdevyy}[1]{\frac{\partial #1}{\partial y}}
\newcommand{\partdevyn}[1]{\frac{\partial^n #1}{\partial y^n}}

% text above = symbol
\newcommand{\overeq}[1]{\ensuremath{\stackrel{#1}=}} 
\newcommand{\greatersmaller}{%
  \mathrel{\ooalign{\raisebox{.6ex}{$>$}\cr\raisebox{-.6ex}{$<$}}}
} % greater and smaller symbols on top of each other, same line

%=-=-=-=-=-=-=-=-=-=-=-=-=-=-=-=-=-=-=-=-=-=-=-=-=-=-=-=-=-=-=-=-=-=-=-=-=-=-=-=-
% User defined macros (non math)

\newcommand{\qedblack}{$\hfill\blacksquare$} % black square end of line
\newcommand{\qedwhite}{\hfill \ensuremath{\Box}} % white square end of line
\newcommand{\hquad}{\hskip0.5em\relax}% half quad space
%\newcommand{\TODO}{\textcolor{red}{\bf TODO!}\;}

\newcommand{\TODO}[1][]{%
    \ifthenelse{\equal{#1}{}}{\textcolor{red}{\bf TODO!}\;}{\textcolor{red}{\textbf {TODO:} #1}\; }%
}
\newcommand{\B}[1]{\textbf{\textup{#1}}} % bold and upright
\renewcommand{\labelitemi}{\scriptsize$\textcolor{DarkPink}{\blacksquare}$} % itemize - squares instead of bullets
\newcommand{\emphasis}[1]{\textls{#1}}

\LetLtxMacro{\originaleqref}{\eqref}
\renewcommand{\eqref}{Eq.~\originaleqref}
\renewcommand*{\eqref}[1]{Eq.~\originaleqref{#1}}





% background images
%%%%%%%
\newcommand\BackgroundPic{%
\put(0,0){%
\parbox[b][\paperheight]{\paperwidth}{%
\vfill
%\centering
\includegraphics[width=0.125\paperwidth,height=\paperheight,%
]{img/background_02.png}% use ,keepaspectratio
\vfill
}}}
%%%%%%%
% end of background image
%%%%%%%%%%%%%% my own frame
\newmdenv[topline=false,bottomline=false]{leftrightbox}
%%%%%%%%%%%%% end
%%%%%%%%%%%%% my own comment
\newcommand{\mycomment}[1]{\begin{leftrightbox}\Pointinghand~\textbf{Comment:}~#1 \end{leftrightbox}}
%%%%%%%%%%%%% end
% my custom note https://tex.stackexchange.com/questions/301993/create-custom-note-environment-with-tcolorbox
\newmdenv[
    topline=false,
    bottomline=false,
    rightline=false,
    innerrightmargin=0pt
]{siderule}
\newenvironment{mynote}%
    {\begin{siderule}\textbf{\Pointinghand~Note:}}
    {\end{siderule}}
    
\newenvironment{myquote}%
    {\begin{adjustwidth}{0.4cm}{0.4cm}\faQuoteLeft\ \itshape}
    { \hfill \faQuoteRight  \end{adjustwidth}}
%%%%%%%%%%%%% my own box
\newcommand{\boxone}[1]{\begin{tcolorbox}[colback = LighterPink,colframe=LightPink]
#1
\end{tcolorbox}}
\newcommand{\boxsimple}[1]{\begin{tcolorbox}[
	colback = white,
	colframe=black!70,
	coltitle = black!20,
	title    = {Problem},]
#1
\end{tcolorbox}}
%%%%%%%%%%%%% end

\newcommand{\bigtitle}[1]{\LARGE\textsc{{\textbf{#1}}}\normalsize\vspace{0.25cm}}

\let\oldemptyset\emptyset
\let\emptyset\varnothing
%algorithmic
\algdef{SE}[DOWHILE]{Do}{doWhile}{\algorithmicdo}[1]{\algorithmicwhile\ #1}%

%%% otherwise minted goes off the page
\setmintedinline{breaklines}




\begin{document}
%=-=-=-=-=-=-=-=-=-=-=-=-=-=-=-=-=-=-=-=-=-=-=-=-=-=-=-=-=-=-=-=-=-=-=-=-=-=-=-=-
% GLOBAL STYLES (DOCUMENT SCOPE)
%=-=-=-=-=-=-=-=-=-=-=-=-=-=-=-=-=-=-=-=-=-=-=-=-=-=-=-=-=-=-=-=-=-=-=-=-=-=-=-=-
% caption: Figure 1 -> <bold> Fig. 1 </bold>
\captionsetup[figure]{labelfont={bf},labelformat={default},labelsep=period,name={Fig.}}


%=-=-=-=-=-=-=-=-=-=-=-=-=-=-=-=-=-=-=-=-=-=-=-=-=-=-=-=-=-=-=-=-=-=-=-=-=-=-=-=-
% TITLE PAGE
%=-=-=-=-=-=-=-=-=-=-=-=-=-=-=-=-=-=-=-=-=-=-=-=-=-=-=-=-=-=-=-=-=-=-=-=-=-=-=-=-
%%%%%%%%%%%%%%%%%%%%%%%%%%%%%%%%%%%%%%%%%
% Formal Book Title Page
% LaTeX Template
% Version 2.0 (23/7/17)
%
% This template was downloaded from:
% http://www.LaTeXTemplates.com
%
% Original author:
% Peter Wilson (herries.press@earthlink.net) with modifications by:
% Vel (vel@latextemplates.com)
%
% License:
% CC BY-NC-SA 3.0 (http://creativecommons.org/licenses/by-nc-sa/3.0/)
% 
% This template can be used in one of two ways:
%
% 1) Content can be added at the end of this file just before the \end{document}
% to use this title page as the starting point for your document.
%
% 2) Alternatively, if you already have a document which you wish to add this
% title page to, copy everything between the \begin{document} and
% \end{document} and paste it where you would like the title page in your
% document. You will then need to insert the packages and document 
% configurations into your document carefully making sure you are not loading
% the same package twice and that there are no clashes.
%
%%%%%%%%%%%%%%%%%%%%%%%%%%%%%%%%%%%%%%%%%

%----------------------------------------------------------------------------------------
%	PACKAGES AND OTHER DOCUMENT CONFIGURATIONS
%----------------------------------------------------------------------------------------



%----------------------------------------------------------------------------------------
%	TITLE PAGE
%----------------------------------------------------------------------------------------



\begin{titlepage} % Suppresses headers and footers on the title page

	%------------------------------------------------
	%	Border decorations
	%------------------------------------------------
    \AddToShipoutPictureBG*{
        \begin{tikzpicture}[overlay,remember picture]
            \draw[line width=10pt]
                ($ (current page.north west) + (4pt,-4pt) $)
                rectangle
                ($ (current page.south east) + (4pt,4pt) $);
            \draw[line width=10pt]
                ($ (current page.north west) + (4pt,-4pt) $)
                rectangle
                ($ (current page.south east) + (-4pt,4pt) $);
        \end{tikzpicture}
        \AtTextUpperLeft{%
            \put(185,12){
                \parbox[b][\paperheight]{\paperwidth}{% parbox
            
                    \centering
                    {\transparent{1.0}
                    \setlength{\fboxsep}{0pt}%
                    \setlength{\fboxrule}{2pt}%
                    \fbox{                                
                        \includegraphics[keepaspectratio=false,width=50pt,height=50pt]{img/title/corner.png}
                        }
                    } % transparent
                } % parbox
            } % put
        } % AtTextUpperLeft
        
        \AtTextUpperLeft{%
            \put(185,-762){
                \parbox[b][\paperheight]{\paperwidth}{% parbox
            
                    \centering
                    {\transparent{1.0}
                    \setlength{\fboxsep}{0pt}%
                    \setlength{\fboxrule}{2pt}%
                    \fbox{                                
                        \includegraphics[keepaspectratio=false,width=50pt,height=50pt]{img/title/corner.png}
                        }
                    } % transparent
                } % parbox
            } % put
        } % AtTextUpperLeft
        
         \AtTextUpperLeft{%
            \put(-338,-762){
                \parbox[b][\paperheight]{\paperwidth}{% parbox
            
                    \centering
                    {\transparent{1.0}
                    \setlength{\fboxsep}{0pt}%
                    \setlength{\fboxrule}{2pt}%
                    \fbox{                                
                        \includegraphics[keepaspectratio=false,width=50pt,height=50pt]{img/title/corner.png}
                        }
                    } % transparent
                } % parbox
            } % put
        } % AtTextUpperLeft
        
        \AtTextUpperLeft{%
            \put(-338,12){
                \parbox[b][\paperheight]{\paperwidth}{% parbox
            
                    \centering
                    {\transparent{1.0}
                    \setlength{\fboxsep}{0pt}%
                    \setlength{\fboxrule}{2pt}%
                    \fbox{                                
                        \includegraphics[keepaspectratio=false,width=50pt,height=50pt]{img/title/corner.png}
                        }
                    } % transparent
                } % parbox
            } % put
        } % AtTextUpperLeft
    } %AddToShipoutPictureBG
    
    
   	%------------------------------------------------
	%	Text alignment
	%------------------------------------------------
	\centering % Centre everything on the title page
	
	\scshape % Use small caps for all text on the title page
	
	\vspace*{\baselineskip} % White space at the top of the page
	
	%------------------------------------------------
	%	Title
	%------------------------------------------------
	
	\rule{\textwidth}{1.6pt}\vspace*{-\baselineskip}\vspace*{2pt} % Thick horizontal rule
	\rule{\textwidth}{0.4pt} % Thin horizontal rule
	
	\vspace{0.75\baselineskip} % Whitespace above the title
	
	{\LARGE NOTES ON\\ \Large 3D GEOMETRY FOR COMPUTER VISION\\ \Large AND SLAM} % Title
	
	\vspace{0.75\baselineskip} % Whitespace below the title
	
	\rule{\textwidth}{0.4pt}\vspace*{-\baselineskip}\vspace{3.2pt} % Thin horizontal rule
	\rule{\textwidth}{1.6pt} % Thick horizontal rule
	
	\vspace{2\baselineskip} % Whitespace after the title block
	
	%------------------------------------------------
	%	Subtitle
	%------------------------------------------------
	Contents
	
	\vspace*{3\baselineskip} % Whitespace under the subtitle
	
	Projective Geometry\\
	Image Rectification\\
	SLAM
	
	\vspace*{3\baselineskip} % Whitespace under the subtitle
	
	%------------------------------------------------
	%	Editor(s)
	%------------------------------------------------
	
	By
	
	\vspace{0.5\baselineskip} % Whitespace before the editors
	
	{\normalfont \Large \mintinline{latex}{0xLeo} (\url{github.com/0xleo}) \\} % Editor list
	
	\vspace{0.5\baselineskip} % Whitespace below the editor list
	
	%\textit{The University of California \\ Berkeley} % Editor affiliation
	
	\vfill % Whitespace between editor names and publisher logo
	
	%------------------------------------------------
	%	Publisher
	%------------------------------------------------
	
	
	\vspace{0.3\baselineskip} % Whitespace under the publisher logo
	
	\today % Date
	
	{DRAFT X.YY} % Draft version
	{\\Missing: \ldots}

\end{titlepage}

%----------------------------------------------------------------------------------------
%\maketitle



%=-=-=-=-=-=-=-=-=-=-=-=-=-=-=-=-=-=-=-=-=-=-=-=-=-=-=-=-=-=-=-=-=-=-=-=-=-=-=-=-
% MAIN DOCUMENT
%=-=-=-=-=-=-=-=-=-=-=-=-=-=-=-=-=-=-=-=-=-=-=-=-=-=-=-=-=-=-=-=-=-=-=-=-=-=-=-=-
\newpage
\tableofcontents
\newpage



%------------------------------ New section ------------------------------%
\section{Higher-order functions, decorators, properties}


\subsection{Functions as first-class citizens}

% see https://www.datacamp.com/community/tutorials/decorators-python
Functions in Python are \emphasis{first-class citizens}. According to Popplestone:
\begin{definition}[first-class citizen (FCC)]
A function is a first-class citizen if all of the below are true
\begin{enumerate}
    \item It can be the actual parameters of functions.
    \item It can be returned as results of functions.
    \item It can be the subject of assignment statements.
    \item It can be tested for equality.
\end{enumerate}
\end{definition}
Therefore if we assign a FCC function to a variable, then the variable acquires the function's properties. The example below illustrates this and of course prints \texttt{<b>some text</b>}. Python allows functions to be assigned to variables since both functions and variables are objects.
\begin{minted}{python}
def makebold(text: str):
    return '<b>' + text + '</b>'

bold = makebold # function gets assigned to a variable
print(bold('some text'))
\end{minted}
The assigned variable (i.e. \texttt{bold}) holds a \textit{ready-to-use} (\textit{ready to call}) copy of the function object and it can call it on any arguments supported by the original function (i.e. \texttt{makebold}), acting as a proxy of it.


\subsection{Higher-order functions}
Another important concept is \emphasis{higher-order functions} (HOF). 
\begin{definition}[higher-order function (HOF)]
A first-class citizen function that takes another function as parameter or returns a (callable) function is called higher-order function.
\end{definition}
Therefore HOFs operate on other functions by augmenting them. Inside a HOF, an inner function is typically defined which takes the same parameters as the input function. However this inner function can be augmented. An important thing to note is that the inner function can be embedded with variables local to the HOF. These local variables then become embedded with the inner function. Then the inner function may or may not be return (see the HOF definition).

The advantage of using HOFs is that if we want to call $g(f(x))$, instead of explicitly calling $g(f(x))$ every time, we can define $h(x) = g(f(x))$ and call the latter instead. 

The best way to demonstrate this is with an example, where the aim is to augment the input function by wrapping its return in the \text{<b></b>} HTML tags.

\lstinputlisting[language=python,caption={Wrapping in \texttt{<b></b>} the return of function \texttt{get\_text} (\detokenize{src/decorators/bold_text.py)}.}, label={lst:hof_bold}]{src/decorators/bold_text.py}
The following text gets printed:
\begin{verbatim}
<i><b>The quick brown fox</b></i>
<i><b>jumps over the</b></i>
\end{verbatim}
An important thing to remember is that the arguments passed in the HOF (i.e. \texttt{get\_text}) must be compatible with the arguments of the inner function (i.e. \texttt{inner}). Then supposing we have a new function to augment \texttt{join3} which takes 3 inputs, then \texttt{make\_bold} wouldn't work on it as the latter expects a 2-argument input. Then a new wrapper \texttt{make\_bold3} would be needed, which operates on a 3-argument function. This would lead to code duplication, as listed below:
% bold_text_2_3_args.py
\lstinputlisting[language=python,caption={Augmenting the functions \texttt{join2} and \texttt{join3} (\detokenize{src/decorators/bold_text_2_3_args.py)} with one wrapper for each.}]{src/decorators/bold_text_2_3_args.py}
It prints:
\begin{verbatim}
<b>A quick</b>
<b>brown fox jumps</b>
\end{verbatim}

A neat way to make HOFs more flexible is by allowing them accept functions that take variable number of arguments. In this case, the augmented function \texttt{inner} accepts any number of arguments and unpacks them (\texttt{*args} operator) in the ``augmentee'' function \texttt{func}. Therefore if we wanted \texttt{make\_bold} to operate on functions that take any number of arguments, we could rewrite the previous code as follows:
\lstinputlisting[language=python,caption={Augmenting the functions \texttt{join2} and \texttt{join3} (\detokenize{src/decorators/bold_text_args.py)} with one wrapper for both.}, label={lst:hof_bold_args}]{src/decorators/bold_text_args.py}

That is the idea of decorators in the next section -- to create a wrapper function which accepts some function inputs, augment or combine (i.e. ``decorate'') the inputs without modifying them, and return a new decorated function.




\subsection{Decorators}

Decorators are  a superset (with a grain of salt, I'm not 100\% sure if that's the official definition) to HOFs, in the sense that they can also accept a class as input.
\begin{definition}[Decorator]
A decorator is a function that takes another class/ function as input or returns a function.
\end{definition}
Python offers syntactic sugar for decorators (therefore HOFs as well). Decorators operate on a the function they aim to wrap, e.g. referring to the last example \texttt{make\_bold} operates on \texttt{join2} and \texttt{join3}. However when a decorator is applied, the input function's behaviour changes permanently. The syntax for decorators is as follows:
\begin{minted}{python}
@decorator
def function_to_decorate():
    # do some stuff
\end{minted}
This is equivalent to:
\begin{minted}{python}
function_to_decorate = decorator(function_to_decorate)
\end{minted}

Therefore instead of calling \texttt{function\_to\_decorate()}, \texttt{decorator(function\_to\_decorate())} is called. Applying this syntax to Listing \ref{lst:hof_bold_args}, the code is rewritten as:

\lstinputlisting[language=python,caption={Applying the \texttt{make\_bold} to \texttt{join2} and \texttt{join3} (\detokenize{src/decorators/dec_bold.py)}.}, label={lst:dec_bold}]{src/decorators/dec_bold.py}

Furthermore, multiple decorators can be chained on the same function. Referring to Listing \ref{lst:dec_bold}, if we wanted to make the text bold and then italic, the decorators would be chained as follows:
\begin{minted}{python}
@make_italic
@make_bold
def foo():
    # return some text
\end{minted}
Therefore they are called from the innermost to the outermost decorator. Hence to decorate the return of \texttt{join3} with \texttt{make\_bold} and then \texttt{make\_italic} we could chain them as follows:
\lstinputlisting[language=python,caption={Applying the \texttt{make\_bold} to \texttt{join2} and \texttt{join3} (\detokenize{src/decorators/dec_bold_it.py)}.},  label={lst:dec_bold_it}]{src/decorators/dec_bold_it.py}
It prints:
\begin{verbatim}
<i><b>brown fox jumps</b></i>
\end{verbatim}


\clearpage
\section{Caching}


%=-=-=-=-=-=-=-=-=-=-=-=-=-=-=-=-=-=-=-=-=-=-=-=-=-=-=-=-=-=-=-=-=-=-=-=-=-=-=-=-
% References
%=-=-=-=-=-=-=-=-=-=-=-=-=-=-=-=-=-=-=-=-=-=-=-=-=-=-=-=-=-=-=-=-=-=-=-=-=-=-=-=-
\newpage
\printbibliography



%=-=-=-=-=-=-=-=-=-=-=-=-=-=-=-=-=-=-=-=-=-=-=-=-=-=-=-=-=-=-=-=-=-=-=-=-=-=-=-=-
% Appendices
%=-=-=-=-=-=-=-=-=-=-=-=-=-=-=-=-=-=-=-=-=-=-=-=-=-=-=-=-=-=-=-=-=-=-=-=-=-=-=-=-
\newpage
\appendix

\section{Appendices}

% ------------------------ New appendix ------------------------ %
\newpage
\subsection{Appendix Example}
%\label{app:my_cool_appendix}

\lstinputlisting[language=python,caption={A code listing (\detokenize{src/foo_bar.py)}.}, label=src:mylabel]{src/foo_bar.py}



\end{document}
