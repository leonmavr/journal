\documentclass[a4paper]{article}
\usepackage[utf8]{inputenc}


%=-=-=-=-=-=-=-=-=-=-=-=-=-=-=-=-=-=-=-=-=-=-=-=-=-=-=-=-=-=-=-=-=-=-=-=-=-=-=-=-
% PREAMBLE
%=-=-=-=-=-=-=-=-=-=-=-=-=-=-=-=-=-=-=-=-=-=-=-=-=-=-=-=-=-=-=-=-=-=-=-=-=-=-=-=-

%%%%%%%%%%%%%%%%%%%%%%%%%%%%%%%%%%%%%%%%%%%%%%%%%%%%%%%%%%%%%%%%%%%%%
% Important styling notes
%%
% For now, to include img.jpg in img/path/to/img.jpg, just use:
% path/to/img.jpg - for details see style.tex
%=-=-=-=-=-=-=-=-=-=-=-=-=-=-=-=-=-=-=-=-=-=-=-=-=-=-=-=-=-=-=-=-=-=-=-=-=-=-=-=-
% Packages
%%
%\usepackage{fullpage} % Package to use full page
\usepackage[top=1in,bottom=1in,left=1in,right=1in,heightrounded]{geometry}

\usepackage{parskip}                    % Package to tweak paragraph skipping
\usepackage{amsmath}                    % standard
\usepackage{amssymb}                    % standard - Double R symbol etc.
\usepackage{hyperref}
\usepackage{amsthm}                     % standard - theorem, definition, etc.
\usepackage{multicol}                   % multiple columns for numbering
\usepackage{enumitem}                   % standard - enumerate styles
\usepackage[utf8]{inputenc}
\usepackage{scrextend}                  % indentation
\usepackage{graphicx}                   % standard - add figures
\usepackage{float}                      % standard - figure position, use [H] option
\usepackage{pifont}                     % symbols http://willbenton.com/wb-images/pifont.pdf
                                        % e.g. \ding{51}
\usepackage{gensymb}                    % degree symbol \degree
\usepackage{xcolor}                     % bg color
\hypersetup{
    colorlinks,
    linkcolor={black!50!black},
    citecolor={blue!50!black},
    urlcolor={blue!80!black}
}
\usepackage{framed}                     % bg color
\usepackage[T1]{fontenc}                % small caps
\usepackage{sectsty}                    % headings colour
\usepackage{mathtools}                  % Loads amsmath
\usepackage{amsthm,thmtools,xcolor}     % coloured theorem
\usepackage[toc,page]{appendix}         % reference to appendix
%\usepackage{titlesec}                   % change chapter, section, etc. formats
\usepackage{xifthen}                    % if, else
\usepackage{etoolbox}
% format numbering in theorem, lemma, etc. environment
\AtBeginEnvironment{theorem}{\setlist[enumerate, 1]{font=\upshape,  wide=0.5em, before=\leavevmode}}
\AtBeginEnvironment{lemma}{\setlist[enumerate, 1]{font=\upshape,  wide=0.5em, before=\leavevmode}}
\usepackage[letterspace=150]{microtype} % \textls{<letterspaced text>} % 0 <= letterspace <= 1000, 1000 = M space
\usepackage{letltxmacro}                % renew commands?
\usepackage{minted}                     % package to list code
    % otherwise minted goes off the page
    \setmintedinline{breaklines}
\usepackage{subfig}
\usepackage{eso-pic}                    % title page bg pic
\usepackage{varwidth}
\PassOptionsToPackage{svgnames}{xcolor}
\usepackage{fontawesome}                % \faQuestionCircle
\usepackage{marvosym}                   %\Pointinghand
\usepackage{mdframed}                   % easy outline frames
\usepackage[many]{tcolorbox}            % colour box for theorem styles
\usepackage{array,booktabs,calc} % table figs and text
\usepackage{comment}                    % \begin{comment}
\usepackage{fancyhdr}                   % page headings
\usepackage{mdframed}                   % boxes
\usepackage[backend=biber,sorting=none,style=ieee]{biblatex}
\usepackage{caption}
%%% caption options {
%\DeclareCaptionFont{white}{\color{white}}
\DeclareCaptionFormat{listing}{\colorbox{magenta!30!gray}{\parbox{\textwidth}{#1#2#3}}}
\captionsetup[lstlisting]{format=listing,labelfont={bf,small},textfont=small,skip=-1pt}
%%% }
\addbibresource{bibliography.bib}
\usepackage{url}
\usepackage{textcomp}
\usepackage[makeroom]{cancel}           % crossed symbols - \cancel{}, \bcancel{}, xcancel{}
\usepackage{algorithm}
\usepackage[noend]{algpseudocode}
\usepackage{tikz}
\usetikzlibrary{arrows.meta,positioning,quotes} % arrows and nodes in tikz
\usepackage{marginnote}                 % things in page margin by \marginnote{...}
\usepackage{pgfplots}
\usepackage{pstricks-add,pst-slpe}      % for fancy tikz arrows
%\usepackage{titlesec}                  % title style
\usepackage{lmodern}                    % a font
\usepackage{titletoc}                   % Required for manipulating the table of contents
\usepackage{titlesec}                   % Allows customization of titles
\usepackage{fouriernc}                  % Use the New Century Schoolbook font
\usepackage{booktabs}                   % better tables
\usepackage{stmaryrd }                  % \varoast
\usepackage{listings}                   % code listings
\usepackage{longtable}                  % table across multiple pages
\usepackage{todonotes}                  % TODO bubbles by \todo{...} command
\usepackage{changepage}                 % paragraph margins
\usepackage{tikz}
\usetikzlibrary{calc}
\usepackage{eso-pic}
\usepackage{transparent}
\usepackage[makeroom]{cancel}

%=-=-=-=-=-=-=-=-=-=-=-=-=-=-=-=-=-=-=-=-=-=-=-=-=-=-=-=-=-=-=-=-=-=-=-=-=-=-=-=-
% Colours for various things
%%


\definecolor{shadecolor}{rgb}{1.,0.933,0.96} % bg color, r,g,b <= 1
\definecolor{medium_blue}{RGB}{60,125,190}
\definecolor{dark_blue}{RGB}{25,60,85}
\definecolor{dark_red}{RGB}{77,16,16}
\definecolor{LightPink}{rgb}{0.92.,0.8,0.84} % bg color, r,g,b <= 1
\definecolor{LighterPink}{rgb}{1.,0.94,0.97} % bg color, r,g,b <= 1
\definecolor{LightestPink}{rgb}{1.,0.95,0.99} % bg color, r,g,b <= 1
\definecolor{DarkestPink}{rgb}{0.36, 0.0, 0.18}
\definecolor{DarkerPink}{rgb}{0.41, 0.0, 0.21}
\definecolor{DarkPink}{rgb}{0.55, 0.05, 0.37}
\definecolor{lightestestpink}{RGB}{255,248,252}
\definecolor{codegray}{rgb}{0.5,0.5,0.5}
\definecolor{codegrayblue}{rgb}{0.35,0.35,0.47}



%=-=-=-=-=-=-=-=-=-=-=-=-=-=-=-=-=-=-=-=-=-=-=-=-=-=-=-=-=-=-=-=-=-=-=-=-=-=-=-=-
% Define my own theorem styles
%%

% "base" styles
\declaretheoremstyle[
  headfont=\color{DarkPink}\bfseries,
  bodyfont=\itshape,
]{colored}

\declaretheoremstyle[
  headfont=\color{DarkPink}\bfseries,
  bodyfont=\normalfont,
]{colored_upright}

% theorems (corollaries, etc) themselves, inherit from my style above
% Usage:
% \begin{theorem} \end{theorem}, \begin{lemma} \end{lemma}, ...
\declaretheorem[
	numberwithin=section,
 	style=colored,
	name=\textsc{Theorem},
]{theorem}

\tcolorboxenvironment{theorem}{
  boxrule=0pt,
  boxsep=2pt,
  colback={magenta!25!white},
  colframe=DarkPink,
  enhanced jigsaw, 
  borderline west={2pt}{0pt}{DarkPink},
  sharp corners,
  before skip=5pt,
  after skip=5pt,
  breakable,
  right=0mm % for equations
}

\declaretheorem[
	numberwithin=section,
 	style=colored,
	name=\textsc{Corollary},
]{corollary}

\tcolorboxenvironment{corollary}{
  boxrule=0pt,
  boxsep=1pt,
  colback={magenta!10!white},
  colframe=DarkPink,
  enhanced jigsaw, 
  borderline west={2pt}{0pt}{DarkPink},
  sharp corners,
  before skip=5pt,
  after skip=5pt,
  breakable,
  right=0mm % for equations
}

\declaretheorem[
	numberwithin=section,
	style=colored,
	name=\textsc{Lemma},
]{lemma}

\tcolorboxenvironment{lemma}{
  boxrule=0pt,
  boxsep=1pt,
  colback={magenta!10!white},
  colframe=DarkPink,
  enhanced jigsaw, 
  borderline west={2pt}{0pt}{DarkPink},
  sharp corners,
  before skip=5pt,
  after skip=5pt,
  breakable,
  right=0mm % for equations
}

\declaretheorem[
	numberwithin=section,
	style=colored,
	name=\textsc{Definition},
]{definition}

\tcolorboxenvironment{definition}{
  boxrule=0pt,
  boxsep=1pt,
  colback={magenta!25!white},
  colframe=DarkPink,
  enhanced jigsaw, 
  borderline west={2pt}{0pt}{DarkPink},
  sharp corners,
  before skip=5pt,
  after skip=5pt,
  breakable,
  right=0mm % for equations
}

\declaretheorem[
	numberwithin=section,
  	style=colored,
  	name=\textsc{Example},
]{exmp}

\declaretheorem[
	numberwithin=section,
  	style=colored,
  	name=\textsc{Solution},
]{soln}

%%% code listings
\lstdefinestyle{code1}{
    backgroundcolor=\color{lightestestpink},   
    commentstyle=\color{codegrayblue},
    keywordstyle=\color{DarkerPink},
    numberstyle=\tiny\color{codegray},
    stringstyle=\color{black!40!cyan},
    basicstyle=\small\ttfamily,
    breakatwhitespace=false,
    breaklines=true,        
    captionpos=t,             
    keepspaces=true,        
    numbers=left,           
    numbersep=5pt,
    showspaces=false, 
    showstringspaces=false,
    showtabs=false,
    tabsize=4
}

%%% code listings
\lstdefinestyle{code1}{
    backgroundcolor=\color{lightestestpink},   
    commentstyle=\color{codegrayblue},
    keywordstyle=\color{DarkerPink},
    numberstyle=\tiny\color{codegray},
    stringstyle=\color{black!40!cyan},
    basicstyle=\small\ttfamily,
    breakatwhitespace=false,
    breaklines=true,        
    captionpos=t,             
    keepspaces=true,        
    numbers=left,           
    numbersep=5pt,
    showspaces=false, 
    showstringspaces=false,
    showtabs=false,
    tabsize=4
}


\lstdefinestyle{terminal}{
    backgroundcolor=\color{black!5},   
    commentstyle=\color{codegrayblue},
    keywordstyle=\color{DarkerPink},
    %numberstyle=\tiny\color{codegray},
    stringstyle=\color{black!40!cyan},
    basicstyle=\small\ttfamily,
    numbers=none,
    breakatwhitespace=false,
    breaklines=true,        
    %captionpos=t,             
    keepspaces=true,        
    %numbers=left,           
    %numbersep=5pt,
    showspaces=false, 
    showstringspaces=false,
    showtabs=false,
    tabsize=4
}

\lstset{style=code1}

%=-=-=-=-=-=-=-=-=-=-=-=-=-=-=-=-=-=-=-=-=-=-=-=-=-=-=-=-=-=-=-=-=-=-=-=-=-=-=-=-
% Headers (size, font, colour)
%%

\makeatletter
\renewcommand{\@seccntformat}[1]{\llap{\textcolor{DarkestPink}{\csname the#1\endcsname}\hspace{1em}}}                    
\renewcommand{\section}{\@startsection{section}{1}{\z@}
{-4ex \@plus -1ex \@minus -.4ex}
{1ex \@plus.2ex }
{\normalfont\large\sffamily\bfseries\textcolor{DarkestPink}}}
\renewcommand{\subsection}{\@startsection {subsection}{2}{\z@}
{-3ex \@plus -0.1ex \@minus -.4ex}
{0.5ex \@plus.2ex }
{\normalfont\sffamily\bfseries\textcolor{DarkestPink}}}
\renewcommand{\subsubsection}{\@startsection {subsubsection}{3}{\z@}
{-2ex \@plus -0.1ex \@minus -.2ex}
{.2ex \@plus.2ex }
{\normalfont\small\sffamily\bfseries\textcolor{DarkestPink}}}                        


%=-=-=-=-=-=-=-=-=-=-=-=-=-=-=-=-=-=-=-=-=-=-=-=-=-=-=-=-=-=-=-=-=-=-=-=-=-=-=-=-
% Numberings, counters and spacings
%%
\numberwithin{equation}{section} % section number in eq/s
\setlength{\jot}{7pt} % spacing in split, gathered env/s



%% Custom examples
%% Output - Example 1,2,...
\newcounter{example}
\newenvironment{example}[1][]{\refstepcounter{example}\par\medskip
   \textbf{Example~\theexample. #1} \rmfamily}{\medskip}
%%%%%%%%%%%% End of unused %%%%%%%%%%%%



%=-=-=-=-=-=-=-=-=-=-=-=-=-=-=-=-=-=-=-=-=-=-=-=-=-=-=-=-=-=-=-=-=-=-=-=-=-=-=-=-
% Paths
%%

%=-=-=-=-=-=-=-=-=-=-=-=-=-=-=-=-=-=-=-=-=-=-=-=-=-=-=-=-=-=-=-=-=-=-=-=-=-=-=-=-
% User defined macros (math mode)
%%


% Curly braces under text. Usage: \myunderbrace{upper}{lower}
\newcommand{\myunderbrace}[2]{\mathrlap{\underbrace{\phantom{#1}}_{#2}} #1}
\newcommand{\setR}{\mathbb{R}} % \ouble R
\newcommand{\setRn}{\mathbb{R}^n} %  double R^n
\newcommand{\setN}{\mathbb{N}} % double N
\newcommand{\setZ}{\mathbb{Z}} % double Z
\let\oldemptyset\emptyset
\let\emptyset\varnothing % nice - looking empty set symbol
\newcommand{\fancyN}{\mathcal{N}} % null space
\newcommand{\fancyR}{\mathcal{R}} % range

\newcommand{\ba}{\textbf{a}}
\newcommand{\be}{\textbf{e}}
\newcommand{\bw}{\textbf{w}}
\newcommand{\bx}{\textbf{x}}
\newcommand{\bu}{\textbf{u}}
\newcommand{\bv}{\textbf{v}}
\newcommand{\by}{\textbf{y}}
\newcommand{\bz}{\textbf{z}}
\newcommand{\bb}{\textbf{b}}
\newcommand{\bA}{\textbf{A}}
\newcommand{\bB}{\textbf{B}}
\newcommand{\bC}{\textbf{C}}
\newcommand{\bD}{\textbf{C}}
\newcommand{\bI}{\textbf{I}}
\newcommand{\bM}{\textbf{M}}
\newcommand{\bO}{\textbf{0}}
\newcommand{\bS}{\textbf{S}}
\newcommand{\bX}{\textbf{X}}
\newcommand{\bU}{\textbf{U}}
\newcommand{\bY}{\textbf{Y}}
% double bars as in norm
%\newcommand{\norm}[1] {\left|\left| #1 \right| \right|} 
\newcommand{\norm}[1]{\left\lVert#1\right\rVert}
\renewcommand{\t}{^{\top}}

\newcommand{\mean}[1]{\bar{#1}}
\newcommand{\var}{\sigma^2}

\newcommand{\partdevx}[1]{\frac{\partial #1}{\partial x}}
\newcommand{\partdevt}[1]{\frac{\partial #1}{\partial t}}
\newcommand{\partdevxx}[1]{\frac{\partial #1}{\partial x}}
\newcommand{\partdevxn}[1]{\frac{\partial^n #1}{\partial x^n}}
\newcommand{\partdevy}[1]{\frac{\partial #1}{\partial y}}
\newcommand{\partdevyy}[1]{\frac{\partial #1}{\partial y}}
\newcommand{\partdevyn}[1]{\frac{\partial^n #1}{\partial y^n}}

% text above = symbol
\newcommand{\overeq}[1]{\ensuremath{\stackrel{#1}=}} 
\newcommand{\greatersmaller}{%
  \mathrel{\ooalign{\raisebox{.6ex}{$>$}\cr\raisebox{-.6ex}{$<$}}}
} % greater and smaller symbols on top of each other, same line

%=-=-=-=-=-=-=-=-=-=-=-=-=-=-=-=-=-=-=-=-=-=-=-=-=-=-=-=-=-=-=-=-=-=-=-=-=-=-=-=-
% User defined macros (non math)

\newcommand{\qedblack}{$\hfill\blacksquare$} % black square end of line
\newcommand{\qedwhite}{\hfill \ensuremath{\Box}} % white square end of line
\newcommand{\hquad}{\hskip0.5em\relax}% half quad space
%\newcommand{\TODO}{\textcolor{red}{\bf TODO!}\;}

\newcommand{\TODO}[1][]{%
    \ifthenelse{\equal{#1}{}}{\textcolor{red}{\bf TODO!}\;}{\textcolor{red}{\textbf {TODO:} #1}\; }%
}
\newcommand{\B}[1]{\textbf{\textup{#1}}} % bold and upright
\renewcommand{\labelitemi}{\scriptsize$\textcolor{DarkPink}{\blacksquare}$} % itemize - squares instead of bullets
\newcommand{\emphasis}[1]{\textls{#1}}

\LetLtxMacro{\originaleqref}{\eqref}
\renewcommand{\eqref}{Eq.~\originaleqref}
\renewcommand*{\eqref}[1]{Eq.~\originaleqref{#1}}





% background images
%%%%%%%
\newcommand\BackgroundPic{%
\put(0,0){%
\parbox[b][\paperheight]{\paperwidth}{%
\vfill
%\centering
\includegraphics[width=0.125\paperwidth,height=\paperheight,%
]{img/background_02.png}% use ,keepaspectratio
\vfill
}}}
%%%%%%%
% end of background image
%%%%%%%%%%%%%% my own frame
\newmdenv[topline=false,bottomline=false]{leftrightbox}
%%%%%%%%%%%%% end
%%%%%%%%%%%%% my own comment
\newcommand{\mycomment}[1]{\begin{leftrightbox}\Pointinghand~\textbf{Comment:}~#1 \end{leftrightbox}}
%%%%%%%%%%%%% end
% my custom note https://tex.stackexchange.com/questions/301993/create-custom-note-environment-with-tcolorbox
\newmdenv[
    topline=false,
    bottomline=false,
    rightline=false,
    innerrightmargin=0pt
]{siderule}
\newenvironment{mynote}%
    {\begin{siderule}\textbf{\Pointinghand~Note:}}
    {\end{siderule}}
    
\newenvironment{myquote}%
    {\begin{adjustwidth}{0.4cm}{0.4cm}\faQuoteLeft\ \itshape}
    { \hfill \faQuoteRight  \end{adjustwidth}}
%%%%%%%%%%%%% my own box
\newcommand{\boxone}[1]{\begin{tcolorbox}[colback = LighterPink,colframe=LightPink]
#1
\end{tcolorbox}}
\newcommand{\boxsimple}[1]{\begin{tcolorbox}[
	colback = white,
	colframe=black!70,
	coltitle = black!20,
	title    = {Problem},]
#1
\end{tcolorbox}}
%%%%%%%%%%%%% end

\newcommand{\bigtitle}[1]{\LARGE\textsc{{\textbf{#1}}}\normalsize\vspace{0.25cm}}

\let\oldemptyset\emptyset
\let\emptyset\varnothing
%algorithmic
\algdef{SE}[DOWHILE]{Do}{doWhile}{\algorithmicdo}[1]{\algorithmicwhile\ #1}%

%%% otherwise minted goes off the page
\setmintedinline{breaklines}




\begin{document}
%=-=-=-=-=-=-=-=-=-=-=-=-=-=-=-=-=-=-=-=-=-=-=-=-=-=-=-=-=-=-=-=-=-=-=-=-=-=-=-=-
% GLOBAL STYLES (DOCUMENT SCOPE)
%=-=-=-=-=-=-=-=-=-=-=-=-=-=-=-=-=-=-=-=-=-=-=-=-=-=-=-=-=-=-=-=-=-=-=-=-=-=-=-=-
% caption: Figure 1 -> <bold> Fig. 1 </bold>
\captionsetup[figure]{labelfont={bf},labelformat={default},labelsep=period,name={Fig.}}


%=-=-=-=-=-=-=-=-=-=-=-=-=-=-=-=-=-=-=-=-=-=-=-=-=-=-=-=-=-=-=-=-=-=-=-=-=-=-=-=-
% TITLE PAGE
%=-=-=-=-=-=-=-=-=-=-=-=-=-=-=-=-=-=-=-=-=-=-=-=-=-=-=-=-=-=-=-=-=-=-=-=-=-=-=-=-
%%%%%%%%%%%%%%%%%%%%%%%%%%%%%%%%%%%%%%%%%%
% Formal Book Title Page
% LaTeX Template
% Version 2.0 (23/7/17)
%
% This template was downloaded from:
% http://www.LaTeXTemplates.com
%
% Original author:
% Peter Wilson (herries.press@earthlink.net) with modifications by:
% Vel (vel@latextemplates.com)
%
% License:
% CC BY-NC-SA 3.0 (http://creativecommons.org/licenses/by-nc-sa/3.0/)
% 
% This template can be used in one of two ways:
%
% 1) Content can be added at the end of this file just before the \end{document}
% to use this title page as the starting point for your document.
%
% 2) Alternatively, if you already have a document which you wish to add this
% title page to, copy everything between the \begin{document} and
% \end{document} and paste it where you would like the title page in your
% document. You will then need to insert the packages and document 
% configurations into your document carefully making sure you are not loading
% the same package twice and that there are no clashes.
%
%%%%%%%%%%%%%%%%%%%%%%%%%%%%%%%%%%%%%%%%%

%----------------------------------------------------------------------------------------
%	PACKAGES AND OTHER DOCUMENT CONFIGURATIONS
%----------------------------------------------------------------------------------------



%----------------------------------------------------------------------------------------
%	TITLE PAGE
%----------------------------------------------------------------------------------------



\begin{titlepage} % Suppresses headers and footers on the title page

	%------------------------------------------------
	%	Border decorations
	%------------------------------------------------
    \AddToShipoutPictureBG*{
        \begin{tikzpicture}[overlay,remember picture]
            \draw[line width=10pt]
                ($ (current page.north west) + (4pt,-4pt) $)
                rectangle
                ($ (current page.south east) + (4pt,4pt) $);
            \draw[line width=10pt]
                ($ (current page.north west) + (4pt,-4pt) $)
                rectangle
                ($ (current page.south east) + (-4pt,4pt) $);
        \end{tikzpicture}
        \AtTextUpperLeft{%
            \put(185,12){
                \parbox[b][\paperheight]{\paperwidth}{% parbox
            
                    \centering
                    {\transparent{1.0}
                    \setlength{\fboxsep}{0pt}%
                    \setlength{\fboxrule}{2pt}%
                    \fbox{                                
                        \includegraphics[keepaspectratio=false,width=50pt,height=50pt]{img/title/corner.png}
                        }
                    } % transparent
                } % parbox
            } % put
        } % AtTextUpperLeft
        
        \AtTextUpperLeft{%
            \put(185,-762){
                \parbox[b][\paperheight]{\paperwidth}{% parbox
            
                    \centering
                    {\transparent{1.0}
                    \setlength{\fboxsep}{0pt}%
                    \setlength{\fboxrule}{2pt}%
                    \fbox{                                
                        \includegraphics[keepaspectratio=false,width=50pt,height=50pt]{img/title/corner.png}
                        }
                    } % transparent
                } % parbox
            } % put
        } % AtTextUpperLeft
        
         \AtTextUpperLeft{%
            \put(-338,-762){
                \parbox[b][\paperheight]{\paperwidth}{% parbox
            
                    \centering
                    {\transparent{1.0}
                    \setlength{\fboxsep}{0pt}%
                    \setlength{\fboxrule}{2pt}%
                    \fbox{                                
                        \includegraphics[keepaspectratio=false,width=50pt,height=50pt]{img/title/corner.png}
                        }
                    } % transparent
                } % parbox
            } % put
        } % AtTextUpperLeft
        
        \AtTextUpperLeft{%
            \put(-338,12){
                \parbox[b][\paperheight]{\paperwidth}{% parbox
            
                    \centering
                    {\transparent{1.0}
                    \setlength{\fboxsep}{0pt}%
                    \setlength{\fboxrule}{2pt}%
                    \fbox{                                
                        \includegraphics[keepaspectratio=false,width=50pt,height=50pt]{img/title/corner.png}
                        }
                    } % transparent
                } % parbox
            } % put
        } % AtTextUpperLeft
    } %AddToShipoutPictureBG
    
    
   	%------------------------------------------------
	%	Text alignment
	%------------------------------------------------
	\centering % Centre everything on the title page
	
	\scshape % Use small caps for all text on the title page
	
	\vspace*{\baselineskip} % White space at the top of the page
	
	%------------------------------------------------
	%	Title
	%------------------------------------------------
	
	\rule{\textwidth}{1.6pt}\vspace*{-\baselineskip}\vspace*{2pt} % Thick horizontal rule
	\rule{\textwidth}{0.4pt} % Thin horizontal rule
	
	\vspace{0.75\baselineskip} % Whitespace above the title
	
	{\LARGE NOTES ON\\ \Large 3D GEOMETRY FOR COMPUTER VISION\\ \Large AND SLAM} % Title
	
	\vspace{0.75\baselineskip} % Whitespace below the title
	
	\rule{\textwidth}{0.4pt}\vspace*{-\baselineskip}\vspace{3.2pt} % Thin horizontal rule
	\rule{\textwidth}{1.6pt} % Thick horizontal rule
	
	\vspace{2\baselineskip} % Whitespace after the title block
	
	%------------------------------------------------
	%	Subtitle
	%------------------------------------------------
	Contents
	
	\vspace*{3\baselineskip} % Whitespace under the subtitle
	
	Projective Geometry\\
	Image Rectification\\
	SLAM
	
	\vspace*{3\baselineskip} % Whitespace under the subtitle
	
	%------------------------------------------------
	%	Editor(s)
	%------------------------------------------------
	
	By
	
	\vspace{0.5\baselineskip} % Whitespace before the editors
	
	{\normalfont \Large \mintinline{latex}{0xLeo} (\url{github.com/0xleo}) \\} % Editor list
	
	\vspace{0.5\baselineskip} % Whitespace below the editor list
	
	%\textit{The University of California \\ Berkeley} % Editor affiliation
	
	\vfill % Whitespace between editor names and publisher logo
	
	%------------------------------------------------
	%	Publisher
	%------------------------------------------------
	
	
	\vspace{0.3\baselineskip} % Whitespace under the publisher logo
	
	\today % Date
	
	{DRAFT X.YY} % Draft version
	{\\Missing: \ldots}

\end{titlepage}

%----------------------------------------------------------------------------------------
%\maketitle



%=-=-=-=-=-=-=-=-=-=-=-=-=-=-=-=-=-=-=-=-=-=-=-=-=-=-=-=-=-=-=-=-=-=-=-=-=-=-=-=-
% MAIN DOCUMENT
%=-=-=-=-=-=-=-=-=-=-=-=-=-=-=-=-=-=-=-=-=-=-=-=-=-=-=-=-=-=-=-=-=-=-=-=-=-=-=-=-




%------------------------------ New section ------------------------------%
\bigtitle{Vector Differentiation Rules}
% ref

Derivatives with respect to a vector come up often in a multitude of areas, such as constrained optimisation, adaptive filtering, and machine learning. We begin by defining the simple case of a scalar differentiated by a vector.
\begin{definition}[gradient]
Let $\bx= (x_1,\ldots,x_n)$ be a column vector and let $f(\bx): \; \setR^n \rightarrow \setR$ be a function that maps to a scalar. The derivative of $f$ w.r.t $\bx$, also known as \emphasis{gradient}, is defined as:
\begin{equation}
    \nabla_{\bx}f := \frac{\partial f}{\partial \bx} := 
    \begin{bmatrix}
    \frac{\partial f}{\partial x_1} & \frac{\partial f}{\partial x_2} & \ldots & \frac{\partial f}{\partial x_n}
    \end{bmatrix}\t
    \label{eq:def_gradient}
\end{equation}
\end{definition}
Both notations in \eqref{eq:def_gradient} are acceptable for the gradient
\begin{exmp}
Find the gradient of the function $f(x_1, x_2, x_3) = x_1 + 3x_2 + 2x_3$.
\end{exmp}
\begin{soln}
\[
\nabla_{\bx}f = 
    \begin{bmatrix}
    \frac{\partial  (x_1 + 3x_2 + 2x_3)}{\partial x_1} & \frac{\partial  (x_1 + 3x_2 + 2x_3)}{\partial x_2} &  \frac{\partial ( x_1 + 3x_2 + 2x_3)}{\partial x_3} 
    \end{bmatrix}\t = 
    \begin{bmatrix}
    1 & 3 & 2
    \end{bmatrix}\t
\]
\qed
\end{soln}
We already notice a property of vector differentiation in the example above; if $f(\bx) = \ba\t\bx$, where $\ba\t = \begin{bmatrix} 1 & 3 & 2 \end{bmatrix}$ is a coefficient row vector and $\bx = \begin{bmatrix} x_1 & x_2 & x_3 \end{bmatrix}\t$ a column vector, then $\nabla_{\bx}f = \ba$.

For reference, the derivative of vector w.r.t. a vector is also defined just to highlight its difference with the derivative of a scalar w.r.t. a vector.
\begin{definition}[Jacobian matrix]
% see Lecture%20Note%203%20-%20Introduction%20to%20Vector%20and%20Matrix%20Differentiation%20(1).pdf
Let $\bx \in \setR^m$ and $\textbf{f}: \setR^m \rightarrow \setR^n$ be a function that returns a $n \times 1$ vector, i.e. $\textbf{f}(\bx) = \begin{bmatrix}f_1(\textbf{x}) & \ldots & f_n(\bx)\end{bmatrix}\t$. Then the derivative of vector $\textbf{f}(\bx)$ w.r.t. $\bx$ is called \emphasis{Jacobian matrix} and is defined as \cite{nielsen}
\begin{equation}
    \textbf{J}(\bx) = \frac{\partial \textbf{f}(\bx)}{\partial \bx} = 
    \begin{bmatrix}
        \frac{\partial f_1(\bx)}{\partial x_1} & \frac{\partial f_1(\bx)}{\partial x_2} & \ldots & \frac{\partial f_1(\bx)}{\partial x_m} \\   
        \frac{\partial f_2(\bx)}{\partial x_1} & \frac{\partial f_2(\bx)}{\partial x_2} & \ldots & \frac{\partial f_2(\bx)}{\partial x_m} \\
        \vdots & \vdots & \ddots & \vdots \\
        \frac{\partial f_n(\bx)}{\partial x_1} & \frac{\partial f_n(\bx)}{\partial x_2} & \ldots & \frac{\partial f_n(\bx)}{\partial x_m} \\   
    \end{bmatrix}
    \label{eq:def_jac_matrix}
\end{equation}
\end{definition}
The determinant of the Jacobian matrix is called Jacobian determinant or \emphasis{Jacobian} for short. Therefore each row $k$ contains the derivative of the scalar function $f_k(.)$ with respect to the elements in $\bx$.
\begin{exmp}
% ref https://www.projectrhea.org/rhea/index.php/Jacobian
Compute the Jacobian matrix of the transformation $T(u, v) = \begin{bmatrix} u & v & u^v \end{bmatrix}\t, \quad u > 0$ \cite{projectrhea}. 
\end{exmp}
\begin{soln}
Using the notation from the definition we compute each row at a time.
\[
f_1(u, v) = u \Rightarrow \frac{\partial f_1(u,v)}{\partial u} = 1, \quad \frac{\partial f_1(u, v)}{\partial v} = 0 
\]
\[
f_2(u,v) = v \Rightarrow \frac{\partial f_1(u,v)}{\partial u} = 0, \quad \frac{\partial f_1(u, v)}{\partial v} = 1
\]
\[
f_3(u,v) = u^v \Rightarrow \frac{\partial f_3(u,v)}{\partial u} = vu^{v-1}, \quad \frac{\partial f_3(u, v)}{\partial v} = u^v \ \ln u
\]
\[
\therefore \ \textbf{J}(u, v) = 
\begin{bmatrix}
1 & 0 \\
0 & 1 \\
vu^{v-1} & u^v \ \ln u
\end{bmatrix}
\]
\qed
\end{soln}
Now we can derive and provide the derivatives of some common scalar ((i), (iii), (iv)) and one vector ((ii)) expressions w.r.t. a vector.
\begin{lemma}[vector differentiation basic properties]
Let $\bx, \ba \in \setR^n$ be two column vectors where $\ba$ is not a function of $\bx$ and $\bA\in \setR^{m\times n}$ be a real matrix. Then:
\begin{flalign}
& (i) && \frac{\partial (\ba\t \bx)}{\partial \bx} = \frac{\partial (\bx\t \ba)}{\partial \bx} = \ba & \\
& (ii) && \frac{\partial (\bA\bx)}{\partial \bx} = \bA & \\
& (iii) && \frac{\partial (\bx\t\bA\t)}{\partial \bx} = \bA\t \; , & \text{if} \quad m=n & \\
& (iv) && \frac{\partial (\bx\t \bA \bx)}{\partial \bx} = (\bA + \bA\t)\ \bx \; , & \text{if} \quad m=n & 
\end{flalign}
\end{lemma}
\begin{proof} \quad

(i)
From the dot product's definition:
\[
\frac{\partial (\ba\t \bx)}{\partial \bx}  =
\begin{bmatrix}
\frac{\sum_{i=1}^n a_ix_i}{\partial x_1} \\ \frac{\sum_{i=1}^n a_ix_i}{\partial x_2} \\ \vdots \\ \frac{\sum_{i=1}^n a_ix_i}{\partial x_n}
\end{bmatrix}
=
\begin{bmatrix}
a_1 \\ a_2 \\ \vdots \\ a_n
\end{bmatrix} = \ba
\]
(ii) If we denote $\ba_1\t,\ldots,\ba_m\t$ the rows of $\bA$ expressed as column vectors, then  product $\bA\bx$ is written as:
\[
\bA\bx = 
\begin{bmatrix}
\ba_1\t \bx \\
\ba_2\t \bx \\
\vdots \\
\ba_m\t \bx
\end{bmatrix}
\]
We apply definition \eqref{eq:def_gradient} on each row, since each row is a scalar:
\[
\therefore \; \frac{\partial ( \bA\bx)}{\partial \bx} =
\begin{bmatrix}
\frac{\partial(\ba_1\t\bx)}{\partial \bx}  \stackrel{\text{(\ref{eq:def_gradient}})}{=} \ba_1 \\
\frac{\partial(\ba_2\t\bx)}{\partial \bx} \stackrel{\text{(\ref{eq:def_gradient}})}{=} \ba_2 \\
\vdots  \\
\frac{\partial(\ba_m\t\bx)}{\partial \bx}  \stackrel{\text{(\ref{eq:def_gradient}})}{=} \ba_m \\
\end{bmatrix}
= 
\begin{bmatrix}
\ba_1 \\
\ba_2 \\
\vdots \\
\ba_m
\end{bmatrix}
= \bA
\]

(iii) Left as exercise.

(iv) % shorter: http://www.ams.sunysb.edu/~zhu/ams571/matrixvector.pdf
The $i$-th element of the product $\bA\bx$, which is a vector, is written with the index notation as follows.
\[
(\bA\bx)_i = \sum_{j=1}^n A_{ij}x_{j}
\]
The dot product of $\bx$, $\bA\bx$ is written as:
\[
\bx\t \bA \bx = \sum_{i}^n x_i \sum_{j=1}^n A_{ij}x_{j}
\]
Applying the definition of the gradient (\eqref{eq:def_gradient}) to the dot product:
\begin{align*}
\frac{\partial(\bx\t \bA \bx)}{\partial \bx} &= \frac{\partial(\sum_{i}^n x_i \sum_{j=1}^n A_{ij}x_{j})}{\partial \bx}   \\
&= \begin{bmatrix}
\frac{\partial(\sum_{i=1}^n x_i \sum_{j=1}^n A_{ij}x_{j})}{\partial x_1} & \ldots &
\frac{\partial(\sum_{i=1}^n x_i \sum_{j=1}^n A_{ij}x_{j})}{\partial x_n} \\
\end{bmatrix}\t
\end{align*}
Using the product rule on the first element:
\begin{align*}
\frac{\sum_{i=1}^n x_i \sum_{j=1}^n A_{ij}x_{j}}{\partial x_1} &= \cancelto{\substack{1,\; i=1\\\text{else}\; 0}}{\frac{\partial \sum_{i=1}^n x_i}{\partial x_1}}\sum_{j=1}^n A_{ij}x_{j} \qquad + \qquad \sum_{i=1}^n x_i \cancelto{\substack{A_{ij},\; j=1\\\text{else}\; 0}}{\frac{\partial \sum_{j=1}^n A_{ij}x_{j}}{\partial x_1}} \\
&= \sum_{j=1}^n A_{1j}x_{j} +  \sum_{i=1}^n x_i A_{i1}\\
&= \ba_{1:}\bx + \ba_{:1}\bx = (\ba_{1:} + \ba_{:1}) \bx
\end{align*}
, where $\ba_{1:}$ denotes the first row of $\bA$ and $\ba_{:1}$ its first column (Matlab notation). Applying the result to the remaining indexes $2,\ldots,n$, we can rewrite the gradient as:
\[
\frac{\partial(\bx\t \bA \bx)}{\partial \bx} = 
\begin{bmatrix}
\ba_{1:} + \ba_{:1} & \ldots & \ba_{n:} + \ba_{:n}
\end{bmatrix}
\bx = (\bA + \bA\t)\bx
\]
\end{proof}

Some other properties that can be readily derived from the basic ones are listed below.

\begin{lemma}[vector differentiation follow-up properties]
If $\bA\in \setR^{n\times n}$ is square and symmetric ($\bA = \bA\t$) and $\bx\in \setR^n$, then
\begin{equation}
    \frac{\partial (\bx\t\bA\bx)}{\partial \bx} = 2\bA\bx
\end{equation}
For $\bA=\bI$, we can derive the vector derivative of the squared norm-2:
\begin{equation}
    \frac{\partial (\bx\t\bx)}{\partial \bx} = \frac{\partial \norm{\bx}^2}{\partial \bx} = 2\bx
\end{equation}
\end{lemma}
Finally, we define the chain rule for vector functions as it's expressed in a slightly different order than the chain rule of scalars.
\begin{lemma}[chain rule for vector function]
Let $\bx\in\setR^n$, $\by \in \setR^r$, $\bz \in \setR^m$, where $\bz = \bz(\by)$ and $\by = \by(\bx)$. Then
\begin{equation}
    \frac{\partial \bz}{\partial \bx} = \frac{\partial \by}{\partial \bx} \frac{\partial \bz}{\partial \by}
\end{equation}
\end{lemma}

\begin{proof} \quad \\
% ref http://www.ams.sunysb.edu/~zhu/ams571/matrixvector.pdf
Proof is found in  \cite{weizhu}.
\end{proof}




%=-=-=-=-=-=-=-=-=-=-=-=-=-=-=-=-=-=-=-=-=-=-=-=-=-=-=-=-=-=-=-=-=-=-=-=-=-=-=-=-
% References
%=-=-=-=-=-=-=-=-=-=-=-=-=-=-=-=-=-=-=-=-=-=-=-=-=-=-=-=-=-=-=-=-=-=-=-=-=-=-=-=-
\newpage
\printbibliography





%=-=-=-=-=-=-=-=-=-=-=-=-=-=-=-=-=-=-=-=-=-=-=-=-=-=-=-=-=-=-=-=-=-=-=-=-=-=-=-=-
% Appendices
%=-=-=-=-=-=-=-=-=-=-=-=-=-=-=-=-=-=-=-=-=-=-=-=-=-=-=-=-=-=-=-=-=-=-=-=-=-=-=-=-


\end{document}